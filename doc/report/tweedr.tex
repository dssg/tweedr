\documentclass{article}

\usepackage{fullpage}
\usepackage{multicol}
\usepackage{outline}

\twocolumn


\title{Tweedr: Twitter for Disaster Response}
\author{Zahra Ashktorab \and Chris Brown \and Jit Nandi \and Aron Culotta}


\begin{document}
\maketitle

\section{Introduction}

\begin{outline}
  \item Context
  \item Problem
  \item Solution Overview
\end{outline}



\section{Data}
\begin{outline}
  \item Unlabeled data from different disasters
  \item Labeling for classification (and uniform vs keyword sampling)
  \item Labeling for extraction
  \item Summary statistics (number labeled/unlabeled; number of each class; number by disaster)
\end{outline}



\section{Methods}
\begin{outline}
  \item Classification
  \item Clustering
  \item Extraction
\end{outline}




\section{Experiments}
\begin{outline}
  \item Classification results
    \begin{outline}
      \item overall precision, recall, f1
      \item compared with predicting on unseen disasters
      \item comparison of sLDA and vanilla classifiers
      \item visualize important features (e.g., sLDA graph)
      \item list some exemplary good/bad classifications
    \end{outline}
  \item Clustering results (maybe don't need accuracy, but at least what percent is duplicate)
  \item Extraction
    \begin{outline}
      \item overall precision, recall, f1, confusion matrix
      \item compared with predicting on unseen disasters
      \item visualize important features
      \item list some exemplary good/bad classifications
    \end{outline}
\end{outline}

\begin{table*}[t]
\centering
\begin{tabular}{|c|c|c|c|c|c|c|}
\hline
              & \multicolumn{3}{c|}{All} &                       \multicolumn{3}{c|}{New Disaster}  \\
\hline
{\bf Method}  &  {\bf F1}        &  {\bf Pr}  &  {\bf Re} & {\bf F1}  &  {\bf Pre}  &  {\bf Re}\\
\hline
{\bf sLDA}    &  0.01 $\pm$ 0.10 & 0.01 $\pm$ 0.10 & 0.01 $\pm$ 0.10 & 0.01 $\pm$ 0.10 & 0.01 $\pm$ 0.10 & 0.01 $\pm$ 0.10\\
{\bf SVM}     &       F1         &       Pr        &       Re        &      F1         &       Pre       &       Re \\
{\bf LogReg}  &       F1         &       Pr        &       Re        &      F1         &       Pre       &       Re \\
\hline
\end{tabular}
\caption{Classification results\label{tab.classification_results}}
\end{table*}



\begin{table*}[t]
\centering
\begin{tabular}{|c|c|c|c|c|c|c|}
\hline
               & \multicolumn{3}{c|}{All} &                       \multicolumn{3}{c|}{New Disaster}  \\
\hline
{\bf Features} &  {\bf F1}        &  {\bf Pr}  &  {\bf Re} & {\bf F1}  &  {\bf Pre}  &  {\bf Re}\\
\hline
{\bf All}      &  0.01 $\pm$ 0.10 & 0.01 $\pm$ 0.10 & 0.01 $\pm$ 0.10 & 0.01 $\pm$ 0.10 & 0.01 $\pm$ 0.10 & 0.01 $\pm$ 0.10\\
{\bf feature1} &       F1         &       Pr        &       Re        &      F1         &       Pre       &       Re \\
{\bf feature2} &       F1         &       Pr        &       Re        &      F1         &       Pre       &       Re \\
\hline
\end{tabular}
\caption{Extraction results\label{tab.extraction_results}}
\end{table*}


\begin{table}[t]
\centering
\begin{tabular}{|l|}
\hline
{\bf Correctly identified as damage or casualty}\\
\hline
xxxx\\
\hline
{\bf Incorrectly identified as damage or casualty}\\
\hline
xxxx\\
\hline
\end{tabular}
\end{table}

\section{Related Work}
\begin{itemize}
\item Extracting Information Nuggets from Disaster- Related Messages in Social Media
\item Practical Extraction of Disaster-Relevant Information from Social Media
\item Social Media Data Mining: A Social Network Analysis Of Tweets During The 2010-2011 Australian Floods
\item TweetTracker: An Analysis Tool for Humanitarian and Disaster Relief
\item Natural Language Processing to the Rescue?: Extracting “Situational Awareness” Tweets During Mass Emergency
\end{itemize}



\section{Conclusions and Future Work}
\begin{outline}
  \item Summarize what we did
  \item Mention limitations
  \item Summarize next steps
\end{outline}

\end{document}
